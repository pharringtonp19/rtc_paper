\documentclass[a4paper,12pt]{article}
\usepackage[dvipsnames]{xcolor}
\usepackage{etoolbox} \usepackage[english]{babel} \usepackage[utf8]{inputenc} \usepackage{amsmath} \usepackage{amsfonts} \usepackage{amsthm} \usepackage{graphicx} \usepackage[colorinlistoftodos]{todonotes} \usepackage{amsfonts} \usepackage{bbm} \usepackage{natbib} \usepackage{setspace} \usepackage{enumitem}
\usepackage{pdfsync} \usepackage{xr}
\usepackage[ruled]{algorithm2e}
\usepackage{tikz}
\usepackage{subfig}
\usepackage{authblk} % NEW!!!
\bibliographystyle{econometrica}


\newtheorem{theorem}{Theorem}[section] \newtheorem{proposition}{Proposition}[section] \newtheorem{lemma}{Lemma}[section]

% \newtheorem{definition}{Definition}[section]
\newtheorem{example}{Example} \newtheorem{corollary}{Corollary}[section] \newtheorem{remark}{Remark}[section] \newtheorem{assumption}{Assumption} \newcommand{\citeposs}[1]{\citeauthor{#1}'s \citeyearpar{#1}} \newcommand\fnote[1]{\captionsetup{font=small}\caption*{#1}}

\def\qed{\rule{2mm}{2mm}} \parskip = 1.5ex
\textwidth 7in
\textheight 10 in
\oddsidemargin -0.4 in
\evensidemargin -0.4in
\topmargin -0.7in





\begin{document}


\begin{titlepage}
\title{Regularizing the Forward Pass: Assessing the Right to Counsel at Scale}
%\shortTitle{Short title for running head}

\author{Patrick Power, Shomik Ghosh and Markus Schwedeler}
\date{\today}
\maketitle
\thispagestyle{empty} % makes the title page number not appear
\vspace{-2em}
\begin{abstract}
 Applied microeconomic work involves making tradeoffs -- assessing which issues are first order, and which can potentially be addressed in an appendix or not at all. Based on the deep learning literature of meta-learning and neural ordinary differential equations, and in the language of category theory, we introduce a unified structure that allows one to think through these tradeoffs: the structure generalizes ordinary least squares, allows for nonparametric cluster effects, and is inherently compositional even under regularization. We then use this framework to assess the effectiveness of an initiative,  growing in popularity across the U.S., known as the Right to Counsel (RTC). Aiming to combat the 3.6 million eviction fillings that occur each year in the U.S., the Right to Counsel ensures access to free legal representation for low-income individuals facing eviction. Complimenting the small, but growing Economic literature on the topic, we consider the indirect effects of this policy. Specifically, we consider whether the policy makes it harder for those currently unhoused to find housing. As some have suggested, if the Right to Counsel increases the cost of evicting a tenant,\footnote{Indeed one of the most consistent findings across the literature is the increase in processing time: \cite{cassidy2022effects} writes: ``The number of days between a case filing and a judgment is also significantly longer in the UA zip codes after program implementation.''} landlords might respond by making it harder for low-income individuals to rent a unit in the first place. Using data from the U.S. Department of Housing and Urban Development, and exploiting the staggered roll-out across the state of Connecticut, our initial results point towards an increase in the length of the housing search in response to the policy. This effect points higher for African American women in particular. We caution, though, that our results are extremely preliminary as the implementation of the policy is ongoing.
\vspace{0.2in}\\
\noindent\textbf{Keywords:} deep learning, evictions\\
%\noindent\textbf{JEL Codes:} key1, key2, key3\\
\end{abstract}
\setcounter{page}{1}
\end{titlepage}

%\thispagestyle{empty}

%\pagebreak \newpage


%\oneandhalfspacing

\section{Framework Introduction}
\subsection{Motivation}
Rarely is there a pre-established estimator that addresses most of the issues competing for ``first-order" importance in applied microeconomic studies. Data is  messy -- clusters of individuals receive the same treatment; people drop out of the sample; outcomes get censored; selection into treatment is unknown. Because of this, it can be helpful to have methods that are \textbf{well-targeted} (i.e. address a specific issue) and \textbf{composable} (i.e. the components fit together) so that researchers can adjust their models to their specific context. With this aim in mind, we illustrate that a regularized version of \cite{finn2017model} composed with a regularized neural ODE (\cite{kelly2020learning}) offers a conceptually simple way to adjust one's estimator for the presence of clustered data as well as to flexibly control the hypothesis space of the model. We highlight the usefulness of this approach both in terms of estimating nonparametric conditional expectation functions as well as low dimensional parameters of interest.
\subsection{Preview of Results}
In the language of Category theory, training the model is done in the Kleisi Category while inference occurs in the Category of Sets. Note for visual clarify we assume that composition of functions is of higher precedence than function application.
\begin{align*} 
& \textrm{linearModel} \ \textcolor{blue}{\text{data}} \\
& \textrm{linearModel} \circ \ \textrm{identityMap} \ \textcolor{blue}{\text{data}} \\ 
& \textrm{linearModel} \circ  \big(\textrm{featureMap} \ \textcolor{blue}{\text{data}}\big) \ \textcolor{purple}{\text{params}} \\ 
& \textrm{linearModel} \circ  \big(\textrm{featureMap} \ \textcolor{blue}{\text{data}}\big) \circ \textrm{identityMap} \  \textcolor{purple}{\text{params}} \\ 
& \textrm{linearModel} \circ  \big(\textrm{featureMap} \ \textcolor{blue}{\text{data}}\big) \circ \big(\textrm{clusterMap} \ \textcolor{blue}{\text{data}}\big)  \textcolor{purple}{\text{params}} \\ 
& \textrm{linearModel} >=>  \big(\textrm{featureMap} \ \textcolor{blue}{\text{data}}\big) >=> \big(\textrm{clusterMap} \ \textcolor{blue}{\text{data}}\big)  \textcolor{purple}{\text{params}} \\ 
\end{align*}
\section{Problem}
\subsection{Context}
To keep things simple, we describe our approach in the specific context of cluster-level randomized control trials where we're interested in estimating treatment heterogeneity.\footnote{ Cluster-level randomized control trials are randomized control trials where treatment varies at a level above the unit of interest} Such experiments are common in development, education, and health settings because they are (A) generally easier to implement, (B) better adhere to the potential outcome framework\footnote{Reduce the chance of spillover effects between treated and non-treated individuals.} and perhaps most importantly\footnote{See John Lists's book, `The Voltage Effect` which highlights this importance in great detail} (C) allow us to understand the the effects of scaling the treatment.\footnote{Many large scale studies such as HIE prefer to include many control variables in their regression specification: size of family, age categories, education level, income, self-reported health status, and use of medical care in the year prior to the start of the experiment, kind of insurance (if any) the person had prior to the experiment, whether family members grew up in a city, suburb, or town, and spending on medical care and dental care prior to the experiment} With a binary treatment variable, such a problem can be decomposed into two separate problems where the objective function is minimized separately over the treatment and control groups.

\begin{align*}
    \underset{f \in \sigma(X)}{\text{inf}} \ E\big[(Y - f)^2\big]
\end{align*}
 

\subsection{Challenge (\textcolor{blue}{The Tragic Triad})\footnote{The expression "tragic triad" is taken from Gradient Surgery for Multi-Task Learning}}
Under the potential outcome framework, clustered level treatment assignment can be roughly thought of as forming the treatment and controls groups via random clustered sampling. From an estimation standpoint, this poses a few challenges because in each treatment group:
\begin{enumerate}
    \item We observe only a subset of the clusters
    \item The distribution of covariates can differ across clusters
    \item The distribution of outcomes conditional on covariates may differ across clusters
\end{enumerate}

The above issues are perhaps only magnified as we increase the dimensionality of the data

\section{Right to Counsel}
In this paper, the term \textit{The Right to Counsel} refers to a policy initiative which ensures that tenants have access to free legal representation in eviction cases. Unlike criminal cases in the U.S., defendants in an eviction case are not provided with a public attorney. A gap in legal representation therefore exists between landlords and tenants which \cite{collinson2022eviction} has documented to be as large as $95\%-1\%$ in some areas in favor of the landlord.
\subsection{Background \& Motivation}
The 2 million evictions that occur each year across the United States are costly to individuals, landlords, courts, and the general public.\footnote{Eviction number from \cite{gromis2022estimating}. (Individual Costs): \cite{collinson2022eviction} writes, ``We find that eviction causes significant disruptions that are reflected in increases in residential mobility, homelessness, and hospital use''. (Court Costs): As \cite{seron2001impact} notes, legal representation actually might decrease housing court costs as the number of appearances and post judgement motions decline. (General Public): As \cite{desmond2019unaffordable} writes, ``Residential instability often brings about other forms of instability—in families, schools, communities— compromising the life chances of adults and children''} Given the severity of these costs, the multitude of factors which contribute to an eviction filling, and the typical manner in which eviction cases are settled, many in the U.S. believe that free legal counsel should be provided to households facing eviction.\footnote{(Multitude of Causes):  David Ehrens writes in his \href{https://dartmouth.theweektoday.com/article/opinion-support-right-counsel-renters/58185}{letter} to the editor of \href{https://dartmouth.theweektoday.com/}{Dartmouth Week} of ``evictions related to the pandemic, chronic housing supply shortages, inequities in lending, generational poverty, and other harms''. (Eviction Proceedings): \textcolor{blue}{Missing Reference} writes ``the vast majority resolved by default or settlement, typically the result of hallway negotiation'' (Growing Interest): \cite{engler2010connecting} writes ``a renewed call for a civil right to counsel, or civil Gideon, has gained momentum \dots as well as a surge in membership in the newly-created National Coalition for a Civil Right to Counsel.''} And indeed, over the past five years, $15$ cities and $3$ states have acted on this belief, initiating a Right to Counsel in some form for low-income households, with additional localities starting pilot studies in the hope of closing the gap in legal representation and improving outcomes. \par 
To some extent, this hope has been empirically justified when looking at the direct legal outcomes of eviction cases. Both in the context of small scale randomized control trials as well as in city-wide roll-outs, researchers have generally found weakly positive to positive results with \cite{seron2001impact}  writing that ``Represented tenants are much less likely to have a final judgment and order of eviction against them'' and \cite{cassidy2022effects} reporting ``Tenants with lawyers are considerably less likely to be subject to possessory judgments, face smaller monetary damages, are less likely to have eviction warrants issued against them, and are ultimately less likely to be evicted.''\footnote{\cite{greiner2012limits} examines the outcomes of two small scale, Massachusetts based, randomized control trials and finds a measurable impact of legal representation in only one of the trials.} \par 
A natural concern, though, is that these legal results might be diminished or possibly out-weighted by the associated indirect effects that emerge when the policy is rolled out at scale. That is, when the policy covers a significant fraction of the population such that landlords are incentivized to respond, the net effect may differ substantially from the above reports. One of the most consistent findings in the literature to date is that legal services increase the duration of eviction proceedings. As voiced both in the academic literature as well as in personal conversations with lawyers, it seems likely that some of these costs will be passed on to low-income households, and perhaps in particular, to those who can least bear them.\footnote{(Academic Shifting Costs): \cite{gunn1995eviction} writes, ``By increasing landlords' costs of doing business, legal services attorneys may enrich their clients at the expense of all other similarly situated poor tenants." (Laywer Shifting Costs):  A lawyer who specializes in evictions wrote via email that `The thing to remember is that higher costs for landlords always get passed on to the tenants in some form (higher rent, deposits, fees, etc.), or the property gets sold, thereby reducing inventory and resulting in higher rents.''} Up to this point, though, there has been little to no empirical work on this highly relevant policy question. \par 
\subsection{Approach}
In order to motivate the specific approach of this paper, it is important to highlight why the above concern remains an open question. There are two closely related reasons for this. The first is that the data required to make an empirical assessment of the Right to Counsel at scale is relatively new. As recently as last year, the most attractive approach to answering this research question was via a counterfactual analysis as in \cite{abramson2021welfare}. The second reason is that the adverse effects of the Right to Counsel are likely difficult to measure. Given the informal nature of evictions, -- Mathew Desmond suggests in his New York Times Best Seller, \textit{Evicted}(\cite{desmond2016evicted}), that informal evictions account for $48\%$ of forced moves while formal evictions account for $24\%$ -- it seems reasonable to expect landlords to operate in some informal, or hard to detect way such as by asking for a higher security deposits, requiring additional months of rent upfront or increasing screening standards. It's not been clear, therefore, what type of data exists that might allow researchers to make even a partial assessment of these general equilibrium effects. \par 
This paper takes a ``noisy'' first step towards addressing both of these issues. First, it exploits the ongoing rollout of the policy across the state of Connecticut where, due to supply constraints of legal services, only low-income individuals in certain zip codes currently receive free legal aid. Second it makes use of data from the U.S. Department of Housing and Urban Development which measures both the characteristics of individuals experiencing homelessness (race, gender, family structure) as well as their length of their housing search. Importantly, this data set is restricted to households who don't face significant barriers to housing. That is, households who are thought to require only limited and partial support. The search length of these households (reflected in the \textcolor{blue}{blue} transition arrows in figure \ref{fig:1}) are therefore a key outcome variable for policy makers are likely a strong indication of whether there are adverse effects of the policy at scale. 

Lastly, while no pre-analysis plan accompanies this paper, the only source of heterogeneity explored is the effects of the policy on Black and female tenants. As well documented in the eviction literature, these subgroups share the greatest likelihood and costs of evictions, as \cite{desmond2019unaffordable} writes, ``Low-income women, especially poor black women, are at high risk of eviction'', and \cite{collinson2022eviction} notes that with regards to the costs of evictions, ``We find particularly sharp negative impacts for female and Black tenants, who drive the effects on labor market outcomes, residential mobility, and interactions with homelessness.'' It seems likely, therefore, that if landlords respond in an adverse way to The Right to Counsel, it would be be towards this sub-population in particular. Hence, all regression specifications are fit both over the entire sample and this sub-sample of interest.
\section{Empirical Strategy}
In contrast to many applied microeconomic papers that assess the sensitivity of their results by varying the selection of the controls in their regression models, we follow the framework presented in our accompanying paper \href{https://github.com/pharringtonp19/rfp_paper/blob/main/Regularizing_the_Forward_Pass.pdf}{Regularizing the Forward Pass} which keeps the set of controls fixed and explores how the estimate varies as we expand the function space that we search over.\footnote{If you are familiar with our ongoing work, you will notice that we have not yet applied the full model in this paper. We plan to do so in the next iteration of the paper.}\par 


The following notation defines the key variables used in the estimators defined below. 
\begin{align*}
    &Y_i: \textrm{Acceptable Move-in Date} \mid \textrm{Search Duration}\\
    &X_i: \textrm{Primary Controls: Age, Gender, Race, Family Size}\\ 
    %&P_i: \textrm{Rapid Rehousing Provider}\\ 
    &Z_i: \textrm{Zip Code} \\
    &D_i: \textrm{Treated Zip Code}
\end{align*}
\textbf{Note}: Subscripts on the outcome corresponding to subsets of individuals who are observed in that corresponding time period. 

\subsection{Difference-in-Difference}
We fit the following difference-in-difference estimator. 
\begin{align*}
    \beta_0 &= \mathbb{E}[Y_1 -Y_0 \mid D=1] -  \mathbb{E}[Y_1 - Y_0 \mid D=0] \end{align*}

\subsection{Difference-in-Difference with Controls}
We then add individual level and zip code controls to the regression specification
\begin{align*}
    Y_i &= \alpha _0 + \beta_0 \textrm{Post}_i \times \textrm{Treated}_i + \beta_1  \textrm{Post}_i + \beta_2 \textrm{Treated}_i \\ 
    &\quad + \beta _3X_i + \beta_4 Z_i + \varepsilon_i
\end{align*}

\subsection{Partially Linear Difference-in-Difference}
As suggested by the explicit functional form, the above estimator doesn't have a non-parametric analogue which may make the interpretation suspect or difficult to some.\footnote{To be clear about this point, it's not obvious how to interpret the coefficient of interest as the result of regression residuals on residuals in the spirit of via Frisch-Waugh-Lovell. That is, the estimator does not have the following nonparametric equivalent: 
\begin{align*}
    Y_i - f_{\theta _1}(Y_i) = \beta_0 (D_i-f_{\theta _2}(D_i)) + \varepsilon_i
\end{align*}}
The natural extension to the difference-in-difference estimator such that it has a Frisch-Waugh-Lovell type of interpretation is the following model where we approximate the conditional expectations via neural network-based estimators.

\begin{itemize}
  \item In the post period, $\mathbb{P}_X,\mathbb{P}_{Y|X}$ might differ because of time and because of the policy
\end{itemize}

\begin{align*}
    Y_{it} - \mathbb{E}[Y_{it} | X_i, Z_i] &= \beta _0 \big(D_i - \mathbb{E}[D_{i} | X_{i}]\big)t + \beta_1 \big(D_i - \mathbb{E}[D_{i} \mid X_{i}]\big) + \varepsilon_i
\end{align*}

\subsection{Nonparametric Difference-in-Difference}
A drawback of the above approach is that it makes no correction for the propensity score. 
\begin{align*}
    \beta _0 = \int d\mathbb{P}_X\Big(\big(\mathbb{E}[Y_1 |X,D=1] - \mathbb{E}[Y_0 |X,D=1]\big) -  \big(\mathbb{E}[Y_1 |X,D=0] - \mathbb{E}[Y_0 |X,D=0]\big)\Big)
\end{align*}
\section{Related Literature}
 While this paper certainly engages with various literatures, from statistical discrimination, to applied deep learning, its central aim is to provide additional insight into the effectiveness of the Right to Counsel. \par 
 In line with the recent Economic works on the topic, it does so by offering a partial assessment of the policy ``at scale''. This approach differs significantly from some of the prior randomized control trial studies where only a limited number of judges were involved as in \cite{greiner2012limits} or where only individuals who were thought to likely benefit from legal representation were provided with laywers from private firms working pro bono as in \cite{seron2001impact}.   

In contrast to the recent Economic literature, though, this paper empirically considers the indirect effects of the policy, thereby complementing \cite{cassidy2022effects} which empirically focuses on the direct effects, and \cite{abramson2021welfare} which considers the indirect effects via a counterfactual exercise.
\section{Conclusion}
\bibliographystyle{plainnat}
\bibliography{bibliography.bib}
\end{document} 

